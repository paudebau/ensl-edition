%% -*- coding: iso-safe-unix; mode: latex -*-
%% pdflatex -recorder -shell-escape sample.tex
\documentclass[lecture=thpr]{ensldocument}

% This lecture/exercices sheet, etc specifics

\thislecturesubject{Pour la d{\'e}monstration}
\thislecturesort{TP n$^o1$}
\thislecturedate{Mercredi 23 septembre \the\year}
\thislectureauthor{Philippe Audebaud}
\thislectureshortauthor{Ph. A.}

\usepackage{blindtext} % pour cette demo

\begin{document}

\paragraph{Prolog sample}

\begin{minted}{prolog}
piExp(R, PI) :- dimension(circle, R, CS), dimension(box, R, BS), PI is CS / BS.

dimension(   box, R, PS) :- PS is R * R.
dimension(circle, R, PS) :- dimensionRec(circle, 0, R, PSR), PS is PSR * 4.

dimensionRec( _, X, R, PS) :-
  X > R, PS is 0.
dimensionRec(circle, X, R, PS) :-
  X =< R, line(circle, X, R, P), dimensionRec(circle, X + 1, R, NP), PS is P + NP.

line(circle, X, R, P) :- hypot(X, Y, R), P is Y.

hypot(X, Y, R) :- nonvar(X), nonvar(Y), sqrt(X * X + Y * Y, R).
hypot(X, Y, R) :- nonvar(X), nonvar(R), sqrt(R * R - X * X, Y).
hypot(X, Y, R) :- nonvar(Y), nonvar(R), sqrt(R * R - Y * Y, X).
\end{minted}

\paragraph{C sample}

\begin{minted}{c}
int main() {
    printf("hello, world");
    return 0;
}
\end{minted}

\paragraph{Python sample}

\begin{minted}{python}
class Foo
    def __init__(self):
        pi = Math::PI

@var = "Pi is approx. #{pi}"
\end{minted}

\blinddocument

\end{document}

\endinput
